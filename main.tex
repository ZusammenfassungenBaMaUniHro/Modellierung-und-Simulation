\documentclass[11pt, fleqn, a4paper, leqno]{scrartcl} %A4
\usepackage[utf8x]{inputenc} %Eingabe
\usepackage[T1]{fontenc} %Font

\usepackage[ngerman]{babel} %Trennnung
\usepackage{amsmath} %Mathesysmbole
\usepackage{graphicx} %Grafiken
\usepackage{listings} %Programmcode
\usepackage{tikz} %Grafiken malen
\usepackage{hyperref}
\hypersetup{
    colorlinks=true, %set true if you want colored links
    linktoc=all,     %set to all if you want both sections and subsections linked
    linkcolor=blue,  %choose some color if you want links to stand out
}

\definecolor{mygreen}{rgb}{0,0.6,0}
\definecolor{mygray}{rgb}{0.5,0.5,0.5}
\definecolor{mymauve}{rgb}{0.58,0,0.82}

\lstset{ %
  backgroundcolor=\color{white},   % choose the background color; you must add \usepackage{color} or \usepackage{xcolor}
  basicstyle=\footnotesize,        % the size of the fonts that are used for the code
  breakatwhitespace=false,         % sets if automatic breaks should only happen at whitespace
  breaklines=true,                 % sets automatic line breaking
  captionpos=b,                    % sets the caption-position to bottom
  commentstyle=\color{mygreen},    % comment style
  deletekeywords={...},            % if you want to delete keywords from the given language
  escapeinside={\%*}{*)},          % if you want to add LaTeX within your code
  extendedchars=true,              % lets you use non-ASCII characters; for 8-bits encodings only, does not work with UTF-8
  frame=single,                    % adds a frame around the code
  keepspaces=true,                 % keeps spaces in text, useful for keeping indentation of code (possibly needs columns=flexible)
  keywordstyle=\color{blue},       % keyword style
  language=Octave,                 % the language of the code
  morekeywords={*,...,mean,avg,clone},            % if you want to add more keywords to the set
  numbers=left,                    % where to put the line-numbers; possible values are (none, left, right)
  numbersep=5pt,                   % how far the line-numbers are from the code
  numberstyle=\tiny\color{mygray}, % the style that is used for the line-numbers
  rulecolor=\color{black},         % if not set, the frame-color may be changed on line-breaks within not-black text (e.g. comments (green here))
  showspaces=false,                % show spaces everywhere adding particular underscores; it overrides 'showstringspaces'
  showstringspaces=false,          % underline spaces within strings only
  showtabs=false,                  % show tabs within strings adding particular underscores
  stepnumber=2,                    % the step between two line-numbers. If it's 1, each line will be numbered
  stringstyle=\color{mymauve},     % string literal style
  tabsize=1,                       % sets default tabsize to 2 spaces
  title=\lstname                   % show the filename of files included with \lstinputlisting; also try caption instead of title
}


\title{Zusammenfassung: Modellierung und Simulation}
\author{Andreas Ruscheinski}
\date{}
\begin{document}
\maketitle
\tableofcontents
\section{Einführung}
	\subsection{Vorbetrachtung}
	\begin{itemize}
		\item Experimente mit dem Computer als Labor
		\item Beispiele: Maschinenanordnung in einer Firma optimieren, Flugzeug fliegen lernen 
		\item Warum: Reale Experimente nicht oder nur teilweise durchführbar (Menschenexperimente, Klimaveränderungen, Entwicklung über mehrere Jahrhunderte)
		\item Risiken: falsche Ergebnisse (Modell passt nicht immer in die Reale Welt, Benutzer obliegt die Verantwortung die Ergebnisse zu interpretieren und kritisch zu hinterfragen)
	\end{itemize}
	\subsection{Definitionen}
	\begin{description}
		\item[System:] Ein System besteht aus Eingaben, Ausgaben und dem Verhalten. Man unterscheidet in Umgebung eines Systems und das Innere eines Systems. Die Umgebung gilt als Abgrenzung. Das Innere beschreibt die Zustände, Komponenten und die Relationen der einzelnen Komponenten. Es muss für jedes System die Systemelemente, Kommunikationsstrukturen und der Veränderung von Werten über die Zeit identifiziert werden.
		\item[Modell:] Ein Modell ist eine Abstraktion eines Systems. Wir stellen ein Modell nur zusammen mit einem System und einem Experiment da. \glqq A model (M) for a system (S) and an experiment (E) is anything to which E can be applied to answer questions about (S).\grqq
		\item[Experiment:] Ein Experiment beinhaltet eine klare Fragestellung (Was soll beantwortet werden?). An ein Experiment sind verschiedene Kriterien gestellt z.b Wiederholbarkeit(unter identischen Voraussetzungen sollen identische Experimente zu den identischen Ergebnissen führen). Wichtig ist dabei die Dokumentation des Experimenten indem die Voraussetzungen, Umgebungsparameter und die Ergebnisse fest gehalten werden. \glqq An experiment is the process of extracting data from a system by excerting it through its input.\grqq
		\item[Simulation:] Eine Simulation ist ein Experiment welches auf einem Modell ausgeführt wird. 
		\item[Computer Modell:] Ein durch einen Computer ausführbares Modell, d.h. definiert in einer durch den Computer interpretierbaren Syntax.
		\item[Computersimulation:] Ein durch einen Computer ausführbares Experiment mit einem computerbasierten Modell.
	\end{description}
\section{Modellierung, Modelle und Sichten}
	\subsection{Wie kommt man von einem System zum Modell?}
		Man unterscheidet dabei zwei grundlegende Vorgehensweisen: induktiv und deduktiv. Bei der induktiven Modellbildung wird das Modell ausgehen von Experimente und Beobachten realer Systeme unter Auswertung der Daten gewonnen. Bei der deduktiven Modellbildung wird das Modell ausgehen von Hypothesen, Lehrmeinungen, Gesetze und Prinzipien gewonnen. Modellierung ist also der Prozess das Wissen über ein System anzuordnen. Es werden die gebildeten Modelle noch weiter unterschieden in Gedankenmodelle (ausgedacht), Verbalmodelle (Modelle auf Basis von Texten und Gesprächen bzw. Beschreibungen) und Formal Modelle (Formale Spezifikation des Modelles durch Logik o.ä). Diese Modelle lassen sich noch weiter in 2 große Bereiche unterteilen, Realmodelle (Welche ausgehend von der Realität gebildet werden) und Ikonische Modelle (dienen zur Veranschaulichung von Sachverhalten, erklären diesen jedoch nicht).
	\subsection{Modellentwicklung}
		Die Modellentwicklung wird unter Berücksichtigung von verschieden Fragen durch geführt. Typische Fragen dabei sind: Was sind wichtige Variablen und wie ist deren Beziehung?, Wie sind die Variablen skaliert?, Ist die Dynamik eher kontinuierlich oder diskret?, Welche Rolle spielt Stochastik?, Was ist interessant? Was sind die Elemente eines Systems?. Man unterscheidet dabei die Variablen in qualitativ skalierte Variablen (incl. Constraints) und quantitative Variablen. Qualitative Variablen haben nur eine begrenze Anzahl an Merkmalsausprägungen (Geschlecht, Religion) wobei quantitative Variablen angeben wie viel, wo von existiert.
	\subsection{Sichten auf Systeme}
	\begin{description}
		\item[Warteschlagenmodelle:]Die Vorstellung des Modells beruht auf die Abarbeitung einer Folge von Aufträgen durch n Bedieneinheiten, wobei die Verteilung der Aufträge und Bearbeitungsdauern variieren können. Beispiele: die immer vollen Warteschlangen in der Mensa, am Flughafen oder im Kaufland und warum muss ich trotz Termin beim Arzt immer warten?
		\item[Makromodelle:] Die Vorstellung des Modells beruht auf die Betrachtung der angenommenen Werte der Variablen. Beispiel: Entwicklung der Bevölkerung
		\item[Mikromodelle:] Die Vorstellung des Modells beruht auf eine differenzierte Betrachtung der Entitäten. Beispiel: Eine Menge von Haushalten, und ihr Einkommen
		\item[Agentenmodelle:] Die Vorstellung des Modells beruht, ähnlich den Mikromodellen, auf einer differenzierten Betrachtung wobei jedoch jedes Element sich über die Laufzeit verändern kann und dabei miteinander kommunizieren. Beispiel: Überlebenssimulation auf einer Insel mit verschiedenen Überlebenden.
		\item[Räumliche Modelle:] Die Vorstellung des Modells beruht darauf das alle Entitäten einen Ort haben und sich ggf. Bewegen. Beispiel: Geosimulation, Moleküle
		\item[Mehrebenenmodelle:] Die Vorstellung des Modells beruht darauf das mehrere Organisationsebenen miteinander verbunden sind und jede Ebene eigene Eigenschaften und Verhaltensmuster aufweisen. Beispiel: Simulation des Menschlichen Organismus (Moleküle, Zellen, Zellverbände, Organe, Organismus)
		\item[Mehrskalenmodelle:]  Die Vorstellung des Modells beruht darauf das sich die Elemente verändern, wobei jedoch die Veränderungen weit auseinander liegen. Im allgemeinen stehen räumliche und zeitliche Skalen in Verbindung. Beispiel: Simulation der Plattentektonik
		\item[Strukturdynamische Modelle:] Die Vorstellung des Modells beruht darauf das die Elemente ihr Interaktionsmuster und ggf. ihre Schnittstellen verändern. Beispiel: Simulation einer Serverfarm, Was passiert wenn 2 Server ausfallen?
		\item[Veränderungen:] Die Veränderung der Eigenschaften kann dabei empirisch Ermittelt worden sein, durch eine Formel abgebildet oder durch Wahrscheinlichkeiten modelliert werden.
	\end{description}
	\subsection{Zeitbasen}
		\subsubsection{Grundlegende Zeit Definition}
			\begin{description}
				\item[Physikalische Zeit:] die Zeit die im modellierten System verstreicht z.B die erste Sekunde nach dem Urknall
				\item[Simulationszeit:] die modellierte physikalische Zeit die während einer Berechnung verstreicht (Eigene Interpretation: quasi die Auflösung für einzelne Schritte, ein Schritt berechnet 1ms oder 1ns)
				\item[Wanduhrzeit:] die Zeit die in der realen Welt währed der Berechnung verstreicht (Die Simulation brauch 30d.q)
			\end{description}
		\subsubsection{Wie wird die Zeit in Modellen dargestellt?}
			\begin{description}
				\item[kontinuierlich:] Fließkommazahlen im Rechner, beliebig feine Auflösung
				\item[diskret:] Ganzzahlen im Rechner, Auflösung fest
			\end{description}
		\subsubsection{Kontinuierliche Modelle}
			\begin{description}
				\item[Beschreibung:]
				\item[Zeit:]
				\item[Zustandsraum:]
				\item[Input/Output:]
			\end{description}
		\subsubsection{Diskrete Modelle - schrittweise}
			\begin{description}
				\item[Beschreibung:]
				\item[Zeit:]
				\item[Zustandsraum:]
				\item[Input/Output:]
			\end{description}
		\subsubsection{Diskrete Modelle - Ereignisse}
			\begin{description}
				\item[Beschreibung:]
				\item[Zeit:]
				\item[Zustandsraum:]
				\item[Input/Output:]
			\end{description}
		\subsubsection{Hybride Modelle}
			\begin{description}
				\item[Beschreibung:]
				\item[Zeit:]
				\item[Zustandsraum:]
				\item[Input/Output:]
			\end{description}
\end{document}